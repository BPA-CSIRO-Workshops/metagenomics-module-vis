% Define the module top matter
% This gets used to create the chapter title page
% NOTES:
%  * When multiple people have authored or contributed to the module, simply use a LaTeX line break
%    (a double-backslash: \\) at the end of each person.
%  * If you don't want this information shown on the module chapter page, simply remove the lines
%    within the \setModuleAuthors{} and \setModuleContributions{} environments
\setModuleTitle{Visualisation}
\setModuleAuthors{%
  Paula Moolhuijzen \mailto{paula.moolhuijzen@curtin.edu.au} \\
}
\setModuleContributions{%
  Peter Sterk \mailto{psterki@ebi.org.uk}%
}

% BEGIN: Module Title Page
% This simply uses the above information and creates a module chapter page
% NOTES:
%  * The chapter page will always appear on odd numbered page
\chapter{\moduleTitle}
\newpage
% END: Module Title Page


% BEGIN: KLOs
% Key Learning Outcomes (KLOs) are an important aspect of any learning/training. They provide
% valuable infomation about what the trainee will have learned, what they will be able to do or know
% abouti at the end of the module. Unlike objectives which are more trainer oriented, KOLs are
% focused on the learner.
% At the end of the module, the KLOs can be used to develop criteria for writing an assessment to
% see if the trainees knowledge/skills have improved as a result of the module.
% 
% Search online for information on how to write KLOs. e.g.
% http://www.teaching-learning.utas.edu.au/__data/assets/word_doc/0014/23333/Learning-outcomes-v9.1.doc
\section{Key Learning Outcomes}

After completing this module the trainee should be able to:
\begin{itemize}
  \item Visualise between sample comparisons, using Emperor PCA
  \item Understand the difference between weighted and unweighted analysis
\end{itemize}
% END KLOs

% BEGIN: Resources Used
% This section can be used to describe the tools and data used during the module. It helps to act as
% a future reference to the trainee
\section{Resources You'll be Using}
 
\subsection{Tools Used}
\begin{description}[style=multiline,labelindent=0cm,align=left,leftmargin=0.5cm]
  \item[Emperor ]\hfill\\
  	\url{http://qiime.org/emperor/tutorial_index.html}
\end{description}

\section{Data sets}
 
\begin{description}[style=multiline,labelindent=0cm,align=left,leftmargin=0.5cm]
  \item[Sutton et al. (2013). Impact of Long-Term Diesel Contamination on Soil Microbial Community Structure ]\hfill\\
    \url{http://www.ncbi.nlm.nih.gov/pmc/articles/PMC3553749/pdf/zam619.pdf}
  \item[Fierer et al. (2010). Forensic identification using skin bacterial communities]\hfill\\
    \url{http://www.ncbi.nlm.nih.gov/pmc/articles/PMC2852011/}
\end{description}

\newpage
% END: Resources Used

% BEGIN: Introduction
\section{Introduction}

% To make a paragraph appear as a "note" to the reader, simply wrap it in a "note" environment like
% this:
\begin{note}
Good data quality and sample metadata is important for visualising metagenomics analysis.
\end{note}

For this tutorial we are using Emperor a browser enabled scatter plot visualisation tool. We will be using the Sutton and Fierer data sets for viewing the principal components analysis (PCoA) from the QIIME Beta diversity analysis (16S). 
% END: Introduction

% BEGIN: Beta Diversity computation and plots
\section{Beta Diversity computation and plots}

Beta diversity represents between sample comparisons based on their composition. The output of these comparisons is a square matrix where a “distance” or dissimilarity is calculated between every pair of community samples, reflecting the dissimilarity between those samples. The data distance matrix can be then visualized with analyses such as PCoA and hierarchical clustering. PCoA is a technique that maps the samples in the distance matrix to a set of axes that show the maximum amount of variation explained. The principal coordinates can be plotted in two or three dimensions to provide an intuitive visualization of the data structure to look at differences between the samples, and look for similarities by sample conditions. 

The Beta diversity workflow you ran earlier contained a number of steps: 
\begin{itemize}
  \item Rarifying the OTU table to remove sample heterogeneity. A rarified OTU table should be used so that artificial diversity induced due to different sampling effort is removed. 
  \item Calculating the Beta diversity, the unifrac weighted and unweighted generated principal coorodinates are created here by default.
\end{itemize}

The following are the required options and inputs for Beta diversity analysis:

\begin{lstlisting}
1.	Option -i, OTU table (*.biom)
2.	Option –t, phylogeny tree (*.tre) from pick_de_novo_otus.py
3.	Option –m, the user-defined mapping file 
4.	Option –o, the output directory
5.	Option –e, the number of sequences per sample (sequencing depth) 
\end{lstlisting}

You do not need to run the following command, it may overwrite the pre-computed analysis. This is the command you ran on the Sutton sub dataset.

\begin{lstlisting}
beta_diversity_through_plots.py -i otus/otu_table.biom -m mapping.txt -o wf_bdiv_even122/ -t otus/rep_set.tre -e 122
\end{lstlisting}

% BEGIN: Prepare the environment
\subsection{Prepare the environment}
The data for this practical can be found in the Taxonomy directory on your desktop. Go to the pre-computed Beta diversity analysis in the Taxonomy directory. 

\begin{steps}

Open the Terminal and go to where the data is stored. Investigate the directories available.

\begin{lstlisting}
cd ~/Desktop/Taxonomy/sutton_full_denoised/
ls –lhtr
\end{lstlisting}
\end{steps}

\begin{steps}
Open the following files:
wf_bdiv_even394/unweighted_unifrac_pc.txt and wf_bdiv_even394/weighted_unifrac_pc.txt 
using the ‘less’ command. The option –S with ‘less’ allows you to view files with unwrapped lines, to escape type q (for quit). To scroll left and right, use the arrow keys.
\begin{lstlisting}
less –S wf_bdiv_even394/unweighted_unifrac_pc.txt
less –S wf_bdiv_even394/weighted_unifrac_pc.txt
\end{lstlisting}
\end{steps}

In each file every sample is listed in the first column, and the subsequent columns contain the value for the sample against the noted principal coordinate. At the bottom of each Principal Coordinate column, you will find the eigen value and percent of variation explained by the coordinate. 
Note the major difference between weighted and unweighted analysis is the inclusion of OTU abundance when calculating distances between communities. You should use weighted if the biological question you are trying to ask takes OTU abundance of your groups into consideration. If some samples are forming groups with weighted, then likely the larger or smaller abundances of several OTUs are the primary driving force in PCoA space, but when all OTUs are considered at equal abundance, these differences are lost (unweighted).

Note discrete vs. continuous only has to do with coloring of the points, and not the calculation of UniFrac distances.
% END: Prepare the environment


\subsection{Emperor}
\begin{steps}
We can now run Emperor to prepare the PCoA plots for visualisation.

\begin{lstlisting}
make_emperor.py -i wf_bdiv_even394/unweighted_unifrac_pc.txt -m sutton_mapping_file.txt –b "Source,Condition,Source&&Condition" –o emporer_output_unweighted
 
ls –l emporer_output_unweighted
\end{lstlisting}
\end{steps}

\begin{steps}
Run the last make_emporer.py command with the weighted_unifrac, using the same command. Make sure you rename the output directory to “emporer_output_weighted“. Look in the Emporer output directories and note the index.html file outputs.
\end{steps}

To view the index.html files created from the ‘make_emperor.py’ steps above a browser like Firefox or Chrome can be used. 

\begin{steps}
Important - To view the Emporer plots we need to switch from the VM to your local machine.
Open the Firefox browser and view the pre-computed Emporer directories available at the following URL. \url{http://www.ebi.ac.uk/~sterk/emperor/}
\end{steps}

View the unweighted analysis on the Sutton sub data set:
\begin{steps}
\url{http://www.ebi.ac.uk/~sterk/emperor/sutton_subset_emporer_output_unweighted/}
\end{steps}


\begin{questions}
Check the ‘Colors’ options to view your samples. 
What can you determine from the PCoA plot?
\begin{answer}

\end{answer}
\end{questions}

\begin{information}
To change the colouring scheme click the ‘Colors’ tab. The available metadata categories are sorted in alphabetical order. Extra information on the features can be found at \url{http://qiime.org/emperor/description_index.html}
Try changing the scaling, visibility, label etc. to improve the display.
Note that Emperor generates publication quality figures in scalable vector graphics (SVG) format.
\end{information}

\begin{steps}
Now look at the weighted unifrac analysis, \url{http://www.ebi.ac.uk/~sterk/emperor/sutton_subset_emporer_output_weighted/}

You can compare the sub data analysis against the full-denoised dataset.
\url{ttp://www.ebi.ac.uk/~sterk/emperor/sutton_full_denoised/}
\end{steps}

\begin{questions}
What are the main differences between the weighted and unweighted plots?
\begin{answer}

\end{answer}
\end{questions}

\begin{questions}
Did you notice any differences between the analysis on the sub data set and the full denoised?
\begin{answer}

\end{answer}
\end{questions}

\begin{questions}
What are your final conclusions on the Sutton dataset? 
\begin{answer}

\end{answer}
\end{questions}

\begin{questions}
What did each of the visualisation methods describe?
\begin{answer}
Taxonomy summary plots: 

Alpha diversity plots:

Beta diversity PCoA plots:

\end{answer}
\end{questions}
	
Now that you are familiar with Emperor we will next view the forensic analysis of the Fierer et. al. data. This experiment matches the bacteria on surfaces to the skin-associated bacteria of the individual who touched the surface. 
This study shows that skin-associated bacteria can be readily recovered from surfaces (including single computer keys and computer mice) and that the structure of these communities can be used to differentiate objects handled by different individuals, even if those objects have been left untouched for up to 2 weeks at room temperature.

Please look the information contained in the mapping file and identify those fields that can be plotted. 

\begin{steps}
\begin{lstlisting}
cd ~/Desktop/Taxonomy/fierer/
less –S fierer_mapping_file.txt
\end{lstlisting}
\end{steps}

\begin{questions}
List any fields that can be plotted for this study. What is the remainder of the information for?
\begin{answer}


\end{answer}
\end{questions}

Run Emperor to prepare the PCoA plots for visualisation
\begin{steps}
\begin{lstlisting}
make_emperor.py -i fierer_unweighted_unifrac_pc.txt -m fierer_mapping_file.txt -b \ "HOST_SUBJECT_ID,ENV_FEATURE,HOST_SUBJECT_ID&&ENV_FEATURE” -o emporer_output_unweighted
\end{lstlisting}
\end{steps}

\begin{steps}
Go to \url{http://www.ebi.ac.uk/~sterk/emperor/fierer_emporer_output_unweighted/}
\end{steps}

\begin{questions}
What did you observe about the clustering?
\begin{answer}

\end{answer}
\end{questions}

\begin{questions}
What fields have been selected?
\begin{answer}

\end{answer}
\end{questions}

\begin{note}
The on-line Qiime tutorials are very good and growing in numbers. It is highly recommended checking out in future the many analysis options.
\end{note}

% END: Beta Diversity computation and plots